\chapter{Introduction}
        
\section{Background}
Many problems in computer science and engineering are hard to solve programatically. It may be difficult or costly to develop accurate models and processes. One attempt to solve some of these problems is to use various machine learning techniques. 

If some of the task to model and develop procedures is left to a machine learning process, hopefully less domain knowledge is required by the application developer. 

Example: robot arm movement?

In a similar way, machine learning may be utilised solved many other tasks in the future.

In order to properly make well use of machine learning, one need to understand well how the machine learning process works, what the possibilities are and the limitations on what role it can have.




\section{Problem}
This report will explore the making of a virtual and autonomous racing driver. 

Racing requires several aspects of tactical behaviour. The driver might need to slow down before a curve in order to turn enough. The driver also need to position the car well before curves in order to have a large enough turning radius, to enable higher speeds, and still not drive a too long distance.

The task of the virtual racing driver is to interpret the track and the state of the car in order to control the car as optimally as possible. 




\section{Background Machine Learning?}
or only in method section?




\section{Purpose and Limitation}
Not sure about the focus of this:

This purpose of this paper is to explore the usage and role of the machine learning technique NEAT in the context of solving the racing problem, using little domain knowledge. Furthermore it will discuss machine learning in general for problems with similar types of characteristics.

The paper will focus on some of the key behaviours of racing and not discuss all aspects that may be relevant for a realistic setting. It will also only discuss racing for one single car, and not the aspects of racing where several cars compete at the same time. 





\iffalse

=== DISPOSITION ===

Background
- Hard to pragmatically solve many types of problems. Machine learning solve some problems (examples), may play an important part in their problems (examples?).
- One type of problem: use machine learning to recreate a somewhat known behaviour, some knowledge of the domain. 
- Racing problem
  - car has momentum and limited manoeuvrability
  - tactical, actions have severe consequences
  - sequence of actions, done with precision

[Discuss and decide the focus of the following section]
- Many machine learning techniques not feasible
  - discuss
- Introduction to neat

Purpose 
- The exploration of NEAT in context of the racing problem. How it may be used and what role it may have.
- Discuss what it could mean to machine learning in general.

Limitation
- Complexity of the environment
- Replicating conceptual similarities of the racing behaviour such as: reasonable positioning, maximising speed where possible, taking curves... 
- One car, not the interaction between cars in a racing setting.


\fi
