% Introduce the different experiments
% If experiment refer to each other, possibly also mention that here.

\chapter{Results \& Discussion}
This chapter presents and discusses the results that have been achieved during the project. The chapter has been divided into parts by the experiments performed, in order to present the way in which the project has developed over time and the way in which the AIs behaviour has progressed alongside these experiments.

Dividing the project into experiments has allowed for an easy way to gradually increase the complexity or change parameters in the environment of the problem in a way that is convenient to analyse and draw conclusions on the behaviour of the AI for every different experiment.

The projects that have been performed have all had great impacts on the final results. Even though not all of them present great results in themselves, they have had significant impacts in our understanding of the problem and the validation of our problem solving and modelling.

% Refer to the description in method
% What was the intended result? What did we want to analyse?
% Present the results of the experiment
    % - What did it do?
    % - Fitness
    % - Generations required before results
    % - (Population)
% Analyse the results and the meanings behind them.
    % - Was it a good result? Why/Why not?
    % - What conclusions can be drawn from the results?
    % - How could we modify this experiment in order to get additional meaningful results?

\section{XOR Experiment}
Before any conclusions could be drawn about the other experiments, the validity of the NEAT implementation had to be confirmed. This was done by performing an XOR experiment. The intended result is for the training algorithm to evolve a network with enough hidden neurons to simulate the XOR function, a non-linear function that cannot be performed by a neural network without hidden neurons.

The experiment performed well and successfully approximated the XOR function after XX generations. One can see the fitness in relation to the generation number in Figure X.X.
% TODO: Talk more about the Figure X.X and what tendencies we can see from the results.

% TODO: Write even more here we can have actual results and numbers to talk about
The result from this experiment is definitely a success, and this means that future experiments can be based upon our NEAT implementation, hopefully without encountering to many problems caused by the NEAT implementation. 


\section{Fixed speed Experiment}
% refer to the description in Method
% example: did it help to help in any way
% example: present data
% example: did the hypothesis work

\section{(Existing steering controller)}
% refer to the description in Method
% example: did it help to help in any way
% example: present data
% example: did the hypothesis work

\section{Acc/brake Experiment}
% Same as Existing steering controller?

\section{Shortest Path}
\section{Multiple track}
\section{Alternating tracks}

\section{Concluding discussion}
% Comparison of the results and discussion of the different experiments not mentioned before.
% General question formulations:
% How to help a machine learning algorithm? Opposition between general algorithms and specialised?
%
Ideas (Gabriel, all may not be relevant to discuss, use for "future work"?):

- Are there any properties for when it get stuck or finds good innovations?

- In this project the neural network controlled the car at a low level. Can improvements be achieved if a larger portion of the problem is modeled by the programmer? What is the conflict then between a general and specialised algorithms? How large of a problem can neat solve on its own? Is it easy to provide it with structural knowledge?

- How should a network be evaluated? Trained for some tracks, possibly used for others? How general can a solution be?

- Do the results of this report influence ML in general

- General algorithms vs. specialised. How well can a general algorithm like NEAT perform? How much knowledge are needed to make NEAT to work better vs. using the knowledge in a specialised algorithm?

\iffalse

% Discussion of the behaviour
% Present current results
This chapter presents and discusses the results that have been achieved during the course of the project. We successfully present the car driving around the track. However the lap that is taken by the car is neither optimal in respect to the race lines nor the time that it takes for the car to travel around the track.

The experiments performed during the course of the project have given varying results. Every experiment have had a large impact on how the project has progressed and on the final results. The results for each experiment and the conclusions that could be drawn are presented and discussed in this chapter.

\section{Experiments}
% Introduce this section somehow.

\subsection{Curve Data as input}
% Present and discuss what results this method achieved for us.
% What kind of machine learning techniques did we use? What did they do differently?
% What behaviour did we see? Why is that?
% Is this behaviour similar or identical to a real race-car driver?
% Is it the most optimal route around the track, with regards to lap time?
% Can we expand this solution further? If so, how?
This experiment was the first one that was carried out with any real progress towards the goal. We managed to produce a machine learning algorithm efficient enough to drive the car around the track without crashing, however given some simplifications; The car automatically accelerates in a straight line, up to a certain point where it stops accelerating and keeps the same constant speed. The neural network has the possibility of controlling the amount of breaking and turning the car does, overriding any automatic acceleration that the car does by itself.

This results in a neural network successfully driving the car around the track. However the path taken is not the optimal one. The path starts of by oscillating left and right between the middle of the track. After a few generations of training, the neural network starts to adjust the oscillating such that the sharper curves can be taken with a wider radius. Given even more training the oscillating almost disappears completely, it only remains before and after some curves.

This behaviour is as mentioned of course not optimal, and neither is it one that a human race-car driver would chose to take. It is both longer and more complicated than would be required for simply driving around the track, without optimising for maximal speed or time.

The training algorithm seems to converge towards a simple neural network between all of the training sessions that has been performed. The neural network produced is one with only one connection between an input and an output. The training required in order to make a complete lap, takes no more than a few minutes. These factors leads us to believe that we can increase the complexity quite significantly before the search space has become to large to be solved within a reasonable time. Thus the limit of NEAT with respect to our problem has not been reached yet.


\subsection{Grid Data as input}
% Present and discuss what results this method achieved.
% What are the benefits and downsides to this compared to other approaches?

\iffalse

====== OLD COMMENTS =====
 
General structure for the results of an experiment:
- Shortly describe the experiment and reference to the description in Method
- Result data (See below)
- Stages in the learning development
  - Did it get stuck at some point?
- What the behaviour became
  - Description
  - Image with interesting racing lines
- Analyse
- Compare to other experiments (if feasible)

Result data:
- Training time
 - Number of generations etc.
 - Number of evaluations
- Fitness
- Analysis of behaviour, may be of different aspects
- Settings variables
- Topology of network
 - measurements of different kinds?

\fi
\fi