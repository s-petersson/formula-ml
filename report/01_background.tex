\chapter{Background}

% The goal of this project is to see if it is possible to help the driver finding the optimal race line for a track before the driver would even have to drive around the track. To make the car/computer learn by itself 

Artificial intelligence (AI) has been a large area of research in computer science for a long time. In 1997 AI research had a major breakthrough, when an AI beat the reigning world champion in chess for the first time ever. Following this breakthrough were AIs such as IBMs "Watson" which beat the two leading world champions in Jeopardy, personal assistants in smartphones such as Apples "Siri" and just recently Google presented an AI that beat a top tier player in the classical game Go.

%TODO change structure to below
%and Google's AI that have just recently beaten a top tier player in the classical game Go.

AI keeps proving itself to be an important concept in computer science and is thus one that is of great interest to computer scientists. There are many ways in which to approach AI development and in most real world scenarios a combination between several approaches will most likely be the optimal solution. However one area that has received much attention lately is the combination between machine learning and neural networks. This project will aim to research this concept further in order to broaden our knowledge about this area and to examine whether or not a computer can learn complicated behaviours, not only function approximations through machine learning. % Det kanske är lite skumt att säga "vi skall försöka undersöka machine learning på en djupare nivå och ta reda på detta om det går att göra dittan o dattan", skulle väl inte säga att vi kommer nå några banbrytande slutsatser direkt?

Machine learning in its simplest form, being used to solve a relatively simple problem is not very complicated. However it becomes vastly more complicated when applied to a more complex problem, where a simple rule based solution does not really suffice in order to solve the problem well enough. In order to effectively research machine learning this project will take advantage of this fact. The project will focus around the problem of taking a Formula1 (F1) car as fast as possible around any given F1 circuit, a problem that is very flexible in terms of complexity. This will allow the project to start of with the very basics behind machine learning then by slowly increasing the complexity of the problem we can closely examine the more complex concepts behind machine learning.

% Tvek på om det är värt att ha med detta stycket. Det finns redan ganska bra motiveringar till varför machine learning är vettigt och vad man kan applicera det inom i de första stycket.
% TODO Saknar ett "tryck" på slutet... någon slags avslutande mening!
The potential applications that AI and machine learning can be used within are as one might imagine endless. In this specific project, it's interesting to consider the gaming industry and racing industries for example. Could such an AI be used in racing games? Maybe even for game AIs in general? Could it potentially be of help to the F1 drivers during practice or even for the team engineers? 
% Skulle kanske kunna lära sig något intressant för racing spel, men det är inte där det fokus ligger? Skulle detta kunna vara till hjälp för generell spel ai?

