\chapter{Introduction}
% Introducera Projektet, skapa intresse
% Saker som behöver ingå:
% - Machine learning
% - Racing - Lyft fram att det är ett svårt problem
% - Datorstyrda fordon / autonoma fordon, dvs. varför har vi fört samman ai och bilar

% Introduce why machine learning is important and why it has received so much popularity if the last few years.

We live in a dawning age of autonomous vehicles. During the last several years the development of autonomous cars has progressed immensely. Many car manufacturers are currently testing their autonomous car prototypes in real traffic and we can expect to see such vehicles on the market within the near future. At the heart of these autonomous vehicles is the field of Artificial intelligence (AI), especially in the form of machine learning, algorithms that learn.

The concept of machine learning is to create well-adapted AI systems that require the minimal amount of human design. The algorithms learn to predict the correct answers within a certain problem domain by inferring knowledge from examples or from experience. 
% Mer om machine learning in general

In autonomous cars, machine learning is used to solve tasks such as steering and image analysis. In order to solve control tasks, such as steering and speed management, the system learns to emulate a specific behaviour by conforming itself to a large set of training examples. The training examples consists of pairs of system inputs, such as sensor data, and the corresponding correct control signals. These sets of training data can be gathered by driving the car while recording both the inputs from all the sensors and cameras on the car, and the control actions taken by the driver. This training process require the recording and quality assurance of vast amounts of example data.

%Outline of: INTRODUCERA MÖJLIGHETEN ATT TRÄNA UTAN DATA.
There are processes to train an AI without a vast amount of data, and therefore be able to achieve a specified behaviour without having to mimic an already driven car. These algorithms are called reinforcement learning, and are not limited by the absence of sample data. Instead of sample data, reinforcement learning algorithms evaluate their behaviour through sets of heuristics(CITE?). The AI AlphaGo that Google created and beat the world champion in Go, Lee Sedol, utilised a type of reinforcement learning algorithm\cite{AlphaGo}. As this type of algorithm does not need any sample data that have been gathered from another instance, their behaviour can excel beyond the behaviour of previous sampled data(CITE?). 

One domain in which this concept show some potential is racing. In competitive racing drivers push the physical limits of their cars. The details matter, minute differences in behaviour can accumulate to large time differences. Machine learning could be used to find efficient driving behaviours. There is also the possibility of using machine learning to create the future generation of racing drivers. FIA, the governing body of Formula 1 and many other competitive racing series, has announced a racing series for autonomous cars \cite{roborace}. In the series the teams will use identical cars, except for the software which the teams can modify to gain an advantage. Thus the goal of the competition is to create the most effective autonomous racing system. 
% Fler saker som är relevant med avseende på machine learning inom racing?

% Racing
%propositions of a complete autonomous racing series that originates from the Formula E series (citation needed).

%Artificial intelligence (AI) has been a large area of research in computer science for a long time. In 1997 AI research had a major breakthrough, when an AI beat the reigning world champion in chess for the first time ever. Following this breakthrough were AIs such as IBMs "Watson" which beat the two leading world champions in Jeopardy, personal assistants in smartphones such as Apples "Siri" and just recently Google presented an AI that beat a top tier player in the classical game Go.

%AI keeps proving itself to be an important concept in computer science and is thus one that is of great interest to computer scientists. git are many ways in which to approach AI development and in most real world scenarios a combination between several approaches will most likely be the optimal solution. However one area that has received much attention lately is the combination between machine learning and neural networks. This project will aim to research this concept further in order to broaden our knowledge about this area and to examine whether or not a computer can learn complicated behaviours, not only function approximations through machine learning.

%Machine learning in its simplest form, being used to solve a relatively simple problem is not very complicated. However it becomes vastly more complicated when applied to a more complex problem, where a simple rule based solution does not really suffice in order to solve the problem well enough. In order to effectively research machine learning this project will take advantage of this fact. The project will focus around the problem of taking a Formula1 (F1) car as fast as possible around any given F1 circuit, a problem that is very flexible in terms of complexity. This will allow the project to start of with the very basics behind machine learning then by slowly increasing the complexity of the problem we can closely examine the more complex concepts behind machine learning.

\section{Purpose}
%TODO Second scentence is hard to read/understand
This report will explore the possibilities of teaching an AI to drive in a racing environment using machine learning. The goal is to find a general and effective behaviour from first principles, i.e. without training it to conform to a predefined behaviour. Furthermore the possibility of finding sufficiently complex behaviours to drive optimally, both locally and globally, will be discussed. 

\section{Problem}
% Each of the sections needs to be broadened, but it is a start

% Introducera varför det är intressant att lösa problemet över huvud taget.
    % FIA new racing series
    % Find optimal behaviours for specific cars?
    % Change in optimal behaviours when modifying a car
    % Could potentially be used in games
    
% Introducera varför man vill göra det med AI.
    % Complex problems are hard to solve manually. AI with Machine learning gives us the potential of solving it my modeling it in a good way.
    % 

Autonomously driving a race car around a circuit has many potential applications. In race cars optimal behaviour is relative to the car, thus it is interesting to examine how a specific car should be driven in order to use its full potential. Autonomously finding the optimal behaviour could present useful information to both engineers and drivers. Engineers could be provided with the change in optimal behaviour when parts are being swapped out or when the car is being tuned. Drivers could learn from the optimal behaviour. Racing against the optimal behaviour in a simulator would give them the opportunity to adjust their driving to that specific car.

In theory the optimal racing behaviour is calculable. Given a specific car and track segment there exist an arbitrarily accurate function describing the time required to drive through the section. Given such a function, one could use optimisation methods to find the minimums in time spent. However, due to the complex nature of the problem, the number of variables in the function is huge. Even if aspects such as the physical environment, e.g. temperature, oxygen levels, and wind, is neglected. Just the number of possible paths through a section is infinite. In order to find an optimal behaviour in a reasonable amount of time the problem must be extensively simplified.

[TODO: Could use some sources in this paragraph?]
The complex nature of this problem presents quite a few problems. Solving complex problems like these manually is not easy. Implementing behaviour for every single scenario is not an option, since the number of scenarios is unreasonably large. Machine learning solves this problem by finding a general behaviour that adapts to the scenarios that it is presented with. Instead of manually implementing behaviour for every scenario, a general machine learning algorithm is implemented and presented with general a model of the problem and its environment.

However finding a general model for the problem and its environment can be very complicated. The information that is to be presented to the AI and the actions taken by the AI must be applicable in many different situations. The behaviour should not be optimal on a specific track, bur rather effective in general. This requires the information that the AI is presented with to not let the AI learn track specific behaviour. One way to solve it is to let the AI make local decisions based on local information. This is similar to how humans drive a car. A person can see the shape of the road in front of the car and steer according to that. This is also true in racing, the driver sees the track and may also have a notion of how the track will proceed throughout the following corners.

% Finding a general behaviour also imposes addition restrictions on the problem. Regardless of which method is used to find a behaviour, the information presented to the AI and the actions taken by the AI must be applicable in many different situations. The behaviour should not be optimal on a specific track, but rather effective in general. To achieve this, the behaviour should be independent of circuit length. One way to solve it is to let the AI make local decisions based on local information. This is similar to how humans drive a car. A person can see the shape of the road in front of the car and steer according to that. This is also true in racing, the driver sees the track and may also have a notion of how the track will proceed throughout the following corners.

% BAKA in i andra stycket, introducera konceptet att man kan vara greedy och bara optimera ~3 kurvor frammåt för att få ett globalt optimum. 
%It also depends on how the track/environment looks further down. If there is a curve followed by a straight line, the fastest way is to carry as much speed as possible out from the curve. A curve followed by another curve, would instead have to be calculated to have the best position to the next curve. These are just examples of how complex behaviour a race driver do need to have to drive the car to the limit.

% Lägg till i diskussion om att mängden lösningar är enorm?
% The controls of a car are not discrete values, as pedals and steering are continuous values within an interval. This creates a boundary (citation needed) where algorithms and methods that utilises decision trees and discrete states non applicable to the racing domain. 





\section{Limitation}


% Limitation to time attack, no head to head strategy, no long term strategy, no internal dynamics of car. 
The scope of this project is limited to developing an AI that finds the optimal behaviour of a race car during a time attack lap. Concepts such as racing strategy, head to head battles between cars, overtaking and pit stops will not be considered when evaluating optimal behaviour. The lap that is to be performed assumes that the cars conditions are optimal and that they will not be modified in any way during the lap, thus aspects that affect the prolonged operation of the car such as fuel efficiency, tyre wear or brake temperatures will not be taken into consideration during the project.

% Limit to neuroevolution/neat, we will not compare different algorithms
The problem of finding optimal behaviour can be solved by using many different types of machine learning. Potentially multiple kinds of machine learning could even be used in parallel to streamline the learning process, by breaking down the problem into several parts. However, for time-related reasons, this project will only be using and evaluating one type of machine learning, in order to allow for a deeper analysis.

% Virtual environment / simulation.
The AI will operate within a virtual environment to increase the flexibility of the car and its environment, compared to operating a car in the real world. A virtual environment will allow for turning on and off certain aspects of the simulation, which may give opportunities and insights that operating in the real world would not. It however also requires the existence of a simulator that is complex enough to simulate the car and its environment in a believable manner. The car needs to behave similarly to how it would in the real world. Thus part of this project is to develop a simulator that satisfies these conditions. More specifically it will include concepts such as turning radius, acceleration and deceleration. However it will not include more advanced concepts such as tyre grip that varies with temperature or internal dynamics of the car.

% Specify even more physical aspects, such as no elevation, no temperatures, oxygen levels etc etc. ¨

% The use of a simulation lets us discard the problem of interpreting the environment, ie we can present the ai with the raw data, we dont need no image analysis etc which would be required irl. 


\section{Literature Review}
% Instruction from 
% https://www.reading.ac.uk/internal/studyadvice/StudyResources/Essays/sta-structuringreport.aspx#literature
% The Literature Review… surveys publications (books, journals and sometimes conference papers) on work that has already been done on the topic of your report. It should only include studies that have direct relevance to your research.
% Introduce your review by explaining how you went about finding your materials, and any clear trends in research that have emerged. Group your texts in themes. Write about each theme as a separate section, giving a critical summary of each piece of work, and showing its relevance to your research. Conclude with how the review has informed your research (things you'll be building on, gaps you'll be filling etc).
% Further reading: http://www.rlf.org.uk/resource/what-is-a-literature-review/
