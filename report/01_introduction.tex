\chapter{Introduction}
% Introduce why machine learning is important and why it has received so much popularity if the last few years.


        
\section{Background}


\iffalse
Artificial intelligence (AI) has been a large area of research in computer science for a long time. In 1997 AI research had a major breakthrough, when an AI beat the reigning world champion in chess for the first time ever. Following this breakthrough were AIs such as IBMs "Watson" which beat the two leading world champions in Jeopardy, personal assistants in smartphones such as Apples "Siri" and just recently Google presented an AI that beat a top tier player in the classical game Go.

%TODO change structure to below
%and Google's AI that have just recently beaten a top tier player in the classical game Go.

AI keeps proving itself to be an important concept in computer science and is thus one that is of great interest to computer scientists. There are many ways in which to approach AI development and in most real world scenarios a combination between several approaches will most likely be the optimal solution. However one area that has received much attention lately is the combination between machine learning and neural networks. This project will aim to research this concept further in order to broaden our knowledge about this area and to examine whether or not a computer can learn complicated behaviours, not only function approximations through machine learning. % Det kanske är lite skumt att säga "vi skall försöka undersöka machine learning på en djupare nivå och ta reda på detta om det går att göra dittan o dattan", skulle väl inte säga att vi kommer nå några banbrytande slutsatser direkt?

Machine learning in its simplest form, being used to solve a relatively simple problem is not very complicated. However it becomes vastly more complicated when applied to a more complex problem, where a simple rule based solution does not really suffice in order to solve the problem well enough. In order to effectively research machine learning this project will take advantage of this fact. The project will focus around the problem of taking a Formula1 (F1) car as fast as possible around any given F1 circuit, a problem that is very flexible in terms of complexity. This will allow the project to start of with the very basics behind machine learning then by slowly increasing the complexity of the problem we can closely examine the more complex concepts behind machine learning.

The potential applications that AI and machine learning can be used within are as one might imagine endless. In this specific project, it's interesting to consider the gaming industry and racing industries for example. Could such an AI be used in racing games? Maybe even for game AIs in general? Could it potentially be of help to the F1 drivers during practice or even for the team engineers? 
\fi

\iffalse

% Introduce machine learning and racing (as a sport). Explain how these are related and what is interesting with this relation.
Many problems in computer science and engineering are hard to solve programatically. It may be difficult or costly to develop accurate models and processes. One attempt to solve some of these problems is to use various machine learning techniques. 

If some of the task to model and develop procedures is left to a machine learning process, hopefully less domain knowledge is required by the application developer. 

Example: robot arm movement?

In a similar way, machine learning may be utilised solved many other tasks in the future.

In order to properly make well use of machine learning, one need to understand well how the machine learning process works, what the possibilities are and the limitations on what role it can have.
\fi



\section{Problem}
% Explain what our problem is and why it is interesting to solve.
This report will explore the making of a virtual and autonomous racing driver. 

Racing requires several aspects of tactical behaviour. The driver might need to slow down before a curve in order to turn enough. The driver also need to position the car well before curves in order to have a large enough turning radius, to enable higher speeds, and still not drive a too long distance.

The task of the virtual racing driver is to interpret the track and the state of the car in order to control the car as optimally as possible. 




\section{Purpose \& Limitation}
% See planning report for the content of this section.
Not sure about the focus of this:

This purpose of this paper is to explore the usage and role of the machine learning technique NEAT in the context of solving the racing problem, using little domain knowledge. Furthermore it will discuss machine learning in general for problems with similar types of characteristics.

The paper will focus on some of the key behaviours of racing and not discuss all aspects that may be relevant for a realistic setting. It will also only discuss racing for one single car, and not the aspects of racing where several cars compete at the same time. 





\iffalse

=== DISPOSITION OLD ===

Background
- Hard to pragmatically solve many types of problems. Machine learning solve some problems (examples), may play an important part in their problems (examples?).
- One type of problem: use machine learning to recreate a somewhat known behaviour, some knowledge of the domain. 
- Racing problem
  - car has momentum and limited manoeuvrability
  - tactical, actions have severe consequences
  - sequence of actions, done with precision

[Discuss and decide the focus of the following section]
- Many machine learning techniques not feasible
  - discuss
- Introduction to neat

Purpose 
- The exploration of NEAT in context of the racing problem. How it may be used and what role it may have.
- Discuss what it could mean to machine learning in general.

Limitation
- Complexity of the environment
- Replicating conceptual similarities of the racing behaviour such as: reasonable positioning, maximising speed where possible, taking curves... 
- One car, not the interaction between cars in a racing setting.


\fi
