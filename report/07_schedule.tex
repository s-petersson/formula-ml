% Planeringsrapporten ska tydligt ange ämnet/problemet som kandidatarbetet ska avhandla, samt hur detta ska göras. Följande rubriker och information ska vara med. Notera att följande rubriker skall vara med oavsett om studien är helt litteraturbaserad, innehåller en empirisk undersökning, eller är ett konstruktionsprojekt.

% Tidsplan
% Den här delen av planeringstrapporten beskriver vad som ska göras och när det ska göras. Personer som ska kontaktas bör också stå med här. Datum eller åtminstone veckor då studenterna ska ge delrapporter samt slutgiltiga presentationen ska stå här. Tidsplanen, kommer naturligtvis vara rätt grov i början.

% Det är viktigt att notera att aktiviteterna inom projektet inte kan ske sekventiellt då dessa aktiviteter är beroende av varandra, vilket innebär att ett antal iterationer mellan dem kommer att ske. Endast genom att iterera mellan dem kommer den uppbyggda kunskapen bli utnyttjad på ett bra sätt. Samma tänkande gäller också rapportskrivandet, d.v.s. uppdatering av ett avsnitt kräver att man uppdaterar andra. Rapportskrivande ska därför ske kontinuerligt under hela projektet.

\chapter{Schedule / Time plan}
* What to do and when to do it.
* High resolution/detail a few weeks ahead, less details further ahead.
* Work in iterations and update the time-plan so that it stays relevant.
* Don't be too abstract.
* If we know what to do, don't plan too much, DO.
* If we are not sure about what to do, or how to do it - PLAN.
* IE. focus on planning the areas where we aren't certain of.
 
Insert pert chart.


\section{Requirements}
\begin{enumerate}
  \item Visualise performance
  \item Make the car drive forward
  \item Be able to follow a curve
  \item Be able to drive though a sequence of curves
  \item Optimise time spent though a single curve by ...
    \begin{enumerate}
        \item ... adapting the speed.
        \item ... adapting the car's positioning.
    \end{enumerate}
  \item Optimise time spent though a sequence of curves by ...
    \begin{enumerate}
        \item ... adapting the speed.
        \item ... adapting the car's positioning.
    \end{enumerate}
    \item Apply more advanced physics
    \begin{enumerate}
        \item Drag
        \item Down force
        \item Tyre wear
        \item Fuel consumption
    \end{enumerate}
\end{enumerate}

\section{Milestones}
Literature

Simulation
Visualisation
    Car and Track
    Network
Basic program structure/architecture

Ai algorithm
    Neural network
        Nodes, edges, basic operations
        Save/load to/from file
    Connection to simulation, feedback.
    Simple training algorithm first
    ...
    MarI/O training algorithm - Evolving Neural Networks through Augmenting Topologies. 