
\chapter{Method}

\section{Desired Behaviour}
% What kind of behaviour do we want to find?
% Behavior -> put requirements on the simulation

\section{Simulation}
% Givet vårt beteende, hur evaluerar / analyserar vi saker i simulationen. Förklara vår sandlåda / universumet vi skapar.
% Requirement on simulation -> satifies the behavior

\section{Training Process/Structure????}
% Hur går vi till väga för att träna en AI, hur är AI och träning kopplat till resten av systemet
% Hur ser träningen ut:
% AIn styr bilen i varje simulation step
% Indata
% Slutgiltliga resultatet av simulationen ger feedback till 

\section{NEAT Implementation}
% Setup of neat "Samma som neat pappret"
% Ta upp skilnaden mellan NEAT och fs-NEAT?

\section{Experiments}
% Introduce the different experiment
% Overview
In order to achieve a great level of flexibility in the project, the process have been divided up into a sequence of experiments. These experiments allows for independently changing any part of the implementation such as the simulation, neural network, input/output data, learning algorithms and more. Thus these experiments have allowed for an easy way of gradually increasing the complexity of the problem, as well as providing a convenient way of analysing the behaviour of the neural network and its learning process in different environments.

% Gå igenom metod för varje experiment enskilt. Vad vill vi testa, hur går vi till väga. 
\subsection{XOR Experiment}
\subsection{Fixed speed Experiment}
\subsection{(Existing steering controller)}
% narrow track, controler stay on mid line
% follow a defined racing line (more close to reality than the first variant)
\subsection{Acc/brake Experiment}
\subsection{Shortest Path}
\subsection{Multiple track}
\subsection{Alternating tracks}


\iffalse

=== DISPOSITION ===

Desired behaviour (move racing theory to theory?)

Simulation
- Evaluation of results

Limitation of techniques to Neat (move to introduction?)

(Fixed network, exhaustive search)

Fundamentals of Neat

Usage/variations of Neat
- Starting structure or not
- Recurrent connections
- Population size
- Probabilities

Aspects in our tests
- Features in physics
- Different indata
  - (Grid)
  - Curve data
    - Angles?
    - Checkpoints
    - Sum
    - Mirror
- Different steering
- Different tracks and their properties
- Training process and task
  - Fitness calculation
  - Long single track
  - Several short parts
  - (Short alternating parts. Hopes for modular network. variants... what types of alternating tasks is needed?

Experiments
- Single long track
  - Fixed low speed, only steering
  - Steering and braking
  - Steer, accelerate and braking
- ...

\fi
 