\begin{abstract} %Needs rework and more describing, selling text.


Driving is a complex task and it is difficult to find an optimal behaviour. It is therefore of interest to investigate how well a machine learning algorithm is able to drive a virtual racing car for a hot lap, if it manages to learn the core aspects of competitive driving.

We have used an algorithm called Neuroevolution of Augmenting Topologies (NEAT) to train artificial neural networks. As for results, it managed to learn to position the car well on the track and to some degree learn to adjust the speed properly. Also, some degree of general behaviour has been observed.

One of the outcome of the project is that it helps to model the problem so that NEAT can progress in gradual steps.



%This paper presents a study of how a simulated race car would behave by learning how to drive a predefined track by reinforcement learning. Racing is a complex and continuous domain, with a lot of variables that are useful in different scenarios, and some variables that may be in excess. The paper describes an evolutionary algorithms NEAT that have been adapted to fit the race simulator and domain. It also focuses on search space reduction to speed up learning and computation.


\end{abstract}


